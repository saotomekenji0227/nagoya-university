\documentclass[]{jsarticle}
\usepackage{amsmath}
\usepackage{comment}
\usepackage{here}
\title{モジュール外部仕様書}
\author{}
\date{}
\begin{document}
\maketitle
\section*{カウンタモジュール}
\noindent
システム名:テキスト・フォーマッタ\\
モージュール名:カウンタモジュール\\
作成年月日:2017年6月29日\\
作成者:早乙女献自\\
最終更新:2017年6月29日\\
Version:1.0\\
\\
概要:\\
カウンタモジュールは出力モジュールから呼び出される.\\
カウンタモジュールは行番号モジュール,改ページモジュール,改行モジュール,タブモジュールを呼び出す.
\\
機能・処理:\\
現在出力を行っているページにおける行数,及び文字数を保持する.出力モジュールからは1文字ずつ受け取り,そのまま出力を行うと半角でMOJIMAX文字を越える場合は改行モジュールの関数lineを呼び出す.また,タブが入力された場合タブモジュールの関数tabを呼び出す.更に,ユーザからの入力による改行が入る場合,行番号モジュールの関数lineheadを呼び出す.但し,改行モジュール呼び出し時,及び行番号モジュール呼び出し時に改行するとLINEMAX行を越える場合は先に改ページモジュールの関数pageを呼び出す.また,ユーザから改ページを指定された場合は改ページモジュールの関数pageを呼び出し,行番号モジュールの関数lineheadを呼び出す.カウンタモジュールの処理が終了したらそのまま終了する.\\
\\
インタフェース:\\
void count(char moji)\\
入力と出力:\\
次に出力したい1文字が入力される.\\
出力する必要の有無を出力する.\\
前提:\\
なし.\\
エラー処理:\\
なし.\\
内部構造:\\
・大域変数(型,名前,初期値)\\
・静的変数(型,名前,初期値)\\
  int mojinum 0\\
  int linenum 1\\
・列挙定数:\\
  LINEMAX 60\\
  MOJIMAX 85\\
・ヘッダファイル(ファイル名,構成要素)\\
 - count.h\\
  関数linehead,line,page及びtabのプロトタイプ宣言を含む.\\
 - txtfmt.h\\
  列挙定数LINEMAX,MOJIMAXの定義を含む.\\
Caller:\\
行番号モジュール,改ページモジュール,改行モジュール,タブモジュール\\
Callee:\\
出力モジュール\\
コメント:\\
特になし.
\end{document}
