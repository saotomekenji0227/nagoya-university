\documentclass[]{jsarticle}
\usepackage{amsmath}
\usepackage{comment}
\usepackage{here}
\title{要求仕様書}
\author{}
\date{}
\begin{document}
%\maketitle
\noindent
txtfmt\\\\
名前\\
   txtfmt - テキストをPostScriptの形式に変換して出力する。\\\\
書式\\
   txtfmt [file]\\\\
説明\\
   txtfmtは引数にファイルを指定した場合、指定されたファイルを読み込み、テキストをPostScriptの\\
   形式に変換して標準出力に出力する。指定されたファイルが.cファイルもしくは.hファイルの場合、予\\
   約語を強調する。ファイルを指定しなかった場合、標準入力をPostScriptの形式に変換して標準出力\\
   に出力する。\\
   各ページのヘッダには1行目にファイル名、2行目にページ番号、日付、ユーザ名を出力する。ただし、\\
   ファイル名は半角のみ受け付け、40文字まで出力する。標準入力を入力とした場合、ファイル名は\\
   「stdin」となる。日付とユーザ名は使用PCに依存する。\\
   本文において、1行には半角85文字、1ページには60行まで出力する。また、ユーザの改行に合わせ\\
   て行番号を表示する。制御文字は、空白、タブ、改行、改ページのみを受け付ける。\\\\
オプション\\
   なし\\\\
環境変数\\
   なし\\\\
関連項目\\
   evince(1), lpr(1)\\\\
バグ\\
   現在バグは確認されていない。\\\\
\begin{center}July 27, 2017\end{center}
\begin{flushright}\vspace{-22pt}txtfmt\end{flushright}
\end{document}
