\documentclass[]{jsarticle}
\usepackage{amsmath}
\usepackage{comment}
\usepackage{here}
\title{要求仕様書}
\author{}
\date{}
\begin{document}
\maketitle
\noindent
システム名:テキスト・フォーマッタ\\
作成年月日:2017年6月22日\\
作成者:ソフトA-6\\
最終更新:2017年6月22日\\
Version:1.0\\
\\
概要:\\
日本語を含むテキストファイル及び標準入力をPostScriptの形式に変換する.システムは以下の状態遷移図と状態遷移表で表される状態遷移機械として捉えられる.\\
\\
機能:\\
引数に指定されたファイルから入力を読み込み,テキストをPostScriptの形式に変換して出力する.引数がない場合は標準入力からデータを読む.但し,入力の日本語はEUCのみで書かれているものとする.出力はまずマクロとしてprologue.psを出力する.その後、各ページにおいてヘッダ,本文を出力する.出力先は標準出力とする.\\
各ページのヘッダにはファイル名,ページ番号,日付,ユーザ名をこの順で出力する.ファイル名は1行目に半角40文字まで記述し,41文字以上の場合は40文字目の次に「...」と記述する。但し,標準入力を入力とした場合,「stdin」をファイル名とする.ページ番号,日付,ユーザ名は「,」で区切って2行目に出力する.ページ番号はPage (数字)の形式で記述し,日付は(西暦)/(月(2桁))/(日(2桁))の形式でPCの日付を取得し出力する.ユーザ名についてもPCのものを取得し出力する.\\
本文については,ヘッダの次に1行空行を設け,その後に出力する.1行につき半角85文字,1ページ60行まで出力する.この値を越えた文字,行についてはそれぞれ次の行,ページに出力を行う.但し,全角文字は半角2文字として数える.\\
行番号について,元ファイルの各行に対応した行番号を出力する.但し,本プログラムの都合で改行した行については行番号の出力を行わず,行番号に当たる部分は空白とする.具体的な出力形式としては4桁まで出力出来るよう行番号の間隔をとり,ここに数字を出力し,その後に「: 」を出力する.4桁を越えた場合は下4桁のみ出力する.この場合のみ,出力最上位が0の場合でも0の出力を行う.\\
出力する文字について,入力における制御文字として,スペース,改行,タブ,改ページはPostScriptでも同じ役割をもつ制御文字を充てる.但し,タブはその行における8n文字目まで空白を出力する.nは自然数かつ8nが現在のその行における文字数より大きい値をとる最小の数とする.なおスペース,改行,タブ,改ページ以外の制御文字は破棄する.また,入力がファイル形式かつC言語だった場合に限り,C言語における予約語の強調を行う.但し,コメントアウトについては予約語の強調を行わず,ダブルクォーテーションやシングルクォーテーションで囲われている文字についても強調を行わない.C言語であるかどうかの判定はファイル名の末尾が「.c」もしくは「.h」であるかどうかにより行う.\\
ユーザインターフェース:\\
コマンド名: txtfmt\\
使用方法: txtfmt [ファイル名]\\
オプション: なし\\
例外処理:\\
引数に指定されたファイルが存在しない場合はエラーメッセージを表示してプログラムを出力し,プログラムを終了する.\\
設計のヒント:\\
ファイル形式の場合,Cであるかどうかの判定を本文記述前に行う必要がある.\\
Cでなかった場合は標準出力と同じ処理を行うため、ここを分けて記述する必要はない.\\
コメント:\\
なし\\

\end{document}
