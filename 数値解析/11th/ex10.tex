\documentclass[]{jsarticle}
\usepackage{amsmath}
\usepackage{comment}
\usepackage{here}
\title{第11回数値解析演習}
\author{081531257,早乙女 献自}
\date{}
\begin{document}
\maketitle
課題1 べき乗法による固有値\\
\begin{eqnarray}
\left(
\begin{array}{rrrrrr}
  6&5&4&3&2&1\\
   & & & & &\\
  5&5&4&3&2&1\\
   & & & & &\\
  3&3&3&3&2&1\\
   & & & & &\\
  2&2&2&2&2&1\\
   & & & & &\\
  1&1&1&1&1&1
\end{array}
\right)
\end{eqnarray}

上記行列について、最大固有値$\lambda$をべき乗法で求めた。以下に実行結果を示す。\\
\$: ./a.out\\
lambda=17.206857\\
\\
課題2 1次元のオイラー法:調和振動子1\\
\begin{eqnarray}
\left\{
\begin{array}{l}
\frac{dq(t)}{dt} = p\\
\frac{dp(t)}{dt} = -q
\end{array}
\right.
\end{eqnarray}
上記連立微分方程式をオイラー法で解いた。結果をp-q平面に図示したものを、次ページにしめす。\\
また、以下にエネルギーを求めた結果を示す。\\
\$: ./a.out\\
before E=0.500000\\
after E=0.610695
\end{document}
