\documentclass[]{jsarticle}
\usepackage{amsmath}
\usepackage{comment}
\usepackage{here}
\title{第12回数値解析演習}
\author{081531257,早乙女 献自}
\date{}
\begin{document}
\maketitle
課題1 ホイン法\\
\begin{eqnarray}
\left\{
\begin{array}{l}
  \frac{dS}{dt} = -\beta SI\\\\
  \frac{dI}{dt} = \beta SI-\gamma I\\\\
  \frac{dR}{dt} = \gamma I 
\end{array}
\right.
\nonumber
\end{eqnarray}
上記の連立微分方程式についてその解をホイン法で求めた。\\
但し、$\beta = 0.0015,\gamma = 0.9$とし、$\delta t = 0.01$で求めた。\\
また、初期値は$t = 0$で$S = 1000,I = 1,R = 0$とした。\\
結果をプロットしたグラフを別紙に示す。\\
課題2 Runge-Kutta法\\
\begin{eqnarray}
\left\{
\begin{array}{l}
  \frac{dx}{dt} = -\sigma x + \sigma y\\\\
  \frac{dy}{dt} = -xy + rx - y\\\\
  \frac{dz}{dt} = xy - bz
\end{array}
\right.
\nonumber
\end{eqnarray}
上記の連立微分方程式についてその解をRunge-Kutta法で求めた。\\
但し、$\sigma = 10.0,r = 28.0,b = \frac{8}{3}$とし、$\delta t = 0.001$で求めた。\\
また、初期値は$t = 0$で$x =1,y = 1,z = 1$とした。\\
結果をプロットしたグラフを別紙に示す。\\
\end{document}
