\documentclass[]{jsarticle}
\usepackage{amsmath}
\usepackage{comment}
\usepackage{here}
\title{第12回数値解析演習}
\author{081531257,早乙女 献自}
\date{}
\begin{document}
\maketitle
課題1 偏微分方程式\\
$\frac{\partial^2 \phi}{\partial x^2}+\frac{\partial^2 \phi}{\partial y^2} = -f(x,y)$

上記偏微分方程式を初期条件を与え解いたところ次ページにしめすようなグラフとなった。\\\\
課題2 感想,意見など\\
アルゴリズムや誤差の理論的な導出についてはテキスト等を見ずに説明出きるようになった。\\
演習の難易度については他の演習と比べてアインシュタインの画像に対して補完をかける演習のみ明らかに難易度が高く,時間もかかったので,難易度を落とさないにしても手際よくプログラムを組めば他の演習と同様程度の時間で解ける演習にしてほしいと感じた。\\
演習内容自体は講義で習った近似的解法が実際に使えることが目で見てわかり,おもしろかったです。
\end{document}
