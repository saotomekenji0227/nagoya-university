\documentclass[]{jsarticle}
\usepackage{amsmath}
\title{第二回数値解析演習}
\author{081531257,早乙女 献自}
\date{2017/04/19}
\begin{document}
\maketitle
\section{課題1}
以下に実行結果を示す。\\
(i)実行結果\\
x1=50000.0000000000000000 x2=0.0001000000011118\\
(ii)実行結果\\
x1=50000.0000000000000000 x2=0.0001000000000000\\
以上より、 (i)については小数点以下12桁以降において誤差が発生したが、(ii)では発生しなかった。\\
\section{課題2}
以下に実行結果を示す。\\
そのままだと0.0000000000000000\\
変形すると0.0000000000001581\\\\
なお、式変形は\\
\begin{equation}
  \frac{a+x}-\frac{a}=\frac{x}{\sqrt{a+x}+\sqrt{a}}
\end{equation}
とした。これにより、桁落ちで計算できなかった式が計算できるようになった。\\
\section{課題3}
以下に実行結果を示す。\\
(i)の場合1.6447253\\
(ii)の場合1.6448090\\
π*π/6=1.6449341\\\\
以上より、デクリメントでループを回すと、大きい値と小さい値の加算
が減り、誤差が小さくなった。
\end{document}
