\documentclass[]{jsarticle}
\usepackage{amsmath}
\usepackage{comment}
\usepackage{here}
\title{第10回数値解析演習}
\author{081531257,早乙女 献自}
\date{}
\begin{document}
\maketitle
課題1 Jacobi法、Gauss-Seidel法
\begin{eqnarray}
\left(
\begin{array}{rrr}
  7&-2&1\\
   & & \\
  -1&-5&-2\\
   & & \\
  -2&-1&6
\end{array}
\right)
\left(
\begin{array}{r}
  x_1\\\\x_2\\\\x_3
\end{array}
\right)
 = 
\left(
\begin{array}{r}
  6\\\\3\\\\14
\end{array}
\right)
\nonumber
\end{eqnarray}
\noindent
(1)上記連立方程式似ついて、Jacobi法とGauss-Seidel法の漸化式を導出した。以下に示す。

Jacobi法
\begin{eqnarray}
\left\{
\begin{array}{c}
x_1^{(k+1)}=\frac{1}{7}(6+2x_2^{(k)}-x_3^{(k)})\\\\
x_2^{(k+1)}=\frac{1}{5}(3+x_1^{(k)}-x+2x_3^{(k)})\\\\
x_3^{(k+1)}=\frac{1}{6}(14+2x_1^{(k)}-6x_2^{(k)})
\end{array}
\right.
\nonumber
\end{eqnarray}

Gauss-Seidel法
\begin{eqnarray}
\left\{
\begin{array}{c}
x_1^{(k+1)}=\frac{1}{7}(6+2x_2^{(k)}-x_3^{(k)})\\\\
x_2^{(k+1)}=\frac{1}{5}(3+x_1^{(k+1)}-x+2x_3^{(k)})\\\\
x_3^{(k+1)}=\frac{1}{6}(14+2x_1^{(k+1)}-6x_2^{(k+1)})
\end{array}
\right.
\nonumber
\end{eqnarray}

\noindent
(2)上記で求めた漸化式についてプログラムにより解を求めた。ただし、初期値は$(x_1,x_2,x_3)=(1,1,1)$とした。以下に結果を示す。

Jacobi法
\begin{eqnarray}
\left(
\begin{array}{r}
  x_1\\\\x_2\\\\x_3
\end{array}
\right)
 = 
\left(
\begin{array}{r}
  0.9999998\\\\1.9999995\\\\2.9999996
\end{array}
\right)
\end{eqnarray}
なお、カウント回数はcount=16となった。

Gauss-Seidel法
\begin{eqnarray}
\left(
\begin{array}{r}
  x_1\\\\x_2\\\\x_3
\end{array}
\right)
 = 
\left(
\begin{array}{r}
  0.9999997\\\\1.9999997\\\\2.9999999
\end{array}
\right)
\end{eqnarray}
なお、カウント回数はcount=11となった。

\noindent
(3)解$(x_1,x_2,x_3)=(1,2,3)$との差を反復ごとに出力した。結果を以下に示す。

Jacobi法
\begin{table}[H]
  \begin{tabular}{|c|c|c|c|}\hline
反復回数&$x_1$の誤差&$x_2$の誤差&$x_3$の誤差\\\hline
0&0.0000000&1.000000&2.0000000\\\hline
1&0.0000000&-0.8000000&-0.1666667\\\hline
2&-0.2047619&-0.0666667&-0.1333333\\\hline
3&0.0000000&-0.0942857&-0.0793651\\\hline
4&-0.0156009&-0.0317460&-0.0157143\\\hline
5&-0.0068254&-0.0094059&-0.0104913\\\hline
6&-0.0011886&-0.0055616&-0.0038428\\\hline
7&-0.0010401&-0.0017748&-0.0013231\\\hline
8&-0.0003181&-0.0007373&-0.0006425\\\hline
9&-0.0001189&-0.0003206&-0.0002289\\\hline
10&-0.0000589&-0.0001153&-0.0000931\\\hline
11&-0.0000197&-0.0000490&-0.0000389\\\hline
12&-0.0000084&-0.0000195&-0.0000147\\\hline
13&-0.0000035&-0.0000076&-0.0000061\\\hline
14&-0.0000013&-0.0000031&-0.0000024\\\hline
15&-0.0000005&-0.0000012&-0.0000010\\\hline
16&-0.0000002&-0.0000005&-0.0000004\\\hline
\end{tabular}
\end{table}

Gauss-Seidel法
\begin{table}[H]
  \begin{tabular}{|c|c|c|c|}\hline
反復回数&$x_1$の誤差&$x_2$の誤差&$x_3$の誤差\\\hline
0&0.0000000&1.000000&2.0000000\\\hline
1&0.0000000&-0.8000000&-0.1333333\\\hline
2&-0.2095238&-0.0952381&-0.0857143\\\hline
3&-0.0149660&-0.0372789&-0.0112018\\\hline
4&-0.0090509&-0.0062909&-0.0040654\\\hline
5&-0.0012166&-0.0018695&-0.0007171\\\hline
6&-0.0004317&-0.0003732&-0.0002061\\\hline
7&-0.0000772&-0.0000979&-0.0000420\\\hline
8&-0.0000220&-0.0000212&-0.0000109\\\hline
9&-0.0000045&-0.0000052&-0.0000024\\\hline
10&-0.0000012&-0.0000012&-0.0000006\\\hline
11&-0.0000003&-0.0000003&-0.0000001\\\hline
\end{tabular}
\end{table}

表から分かるとおり、Gauss-Seidel法の方が早く収束した。
\end{document}
