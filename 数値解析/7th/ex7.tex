\documentclass[]{jsarticle}
\usepackage{amsmath}
\title{第7回数値解析演習}
\author{081531257,早乙女 献自}
\date{2017/04/19}
\begin{document}
\maketitle
$$\int_0^1 4\sqrt{1-x^2}dx = \pi$$
\noindent
(1)上記の式について積分と誤差をSimpson則で計算した。その結果を以下に示す。\\
\begin{eqnarray}
T^{(1)}_1=2.9760677434252 誤差0.1655249101646\nonumber\\
T^{(1)}_2=3.0835951549470 誤差0.0579974986428\nonumber\\
T^{(1)}_3=3.1211891697754 誤差0.0204034838144\nonumber\\
T^{(1)}_4=3.1343976689846 誤差0.0071949846052\nonumber\\
T^{(1)}_5=3.1390522178936 誤差0.0025404356962\nonumber\\
T^{(1)}_6=3.1406950760805 誤差0.0008975775093\nonumber\\
T^{(1)}_7=3.1412754189356 誤差0.0003172346542\nonumber\\
T^{(1)}_8=3.1414805131441 誤差0.0001121404457\nonumber\\
T^{(1)}_9=3.1415530093071 誤差0.0000396442827\nonumber\\
T^{(1)}_10=3.1415786378121 誤差0.0000140157777\nonumber\\
T^{(1)}_11=3.1415876983689 誤差0.0000049552209\nonumber\\
T^{(1)}_12=3.1415909016732 誤差0.0000017519166\nonumber
\end{eqnarray}
(2)上記の式についてRomberg積分法で$T_4^{(4)}$が得られるまで計算した。以下にその結果と誤差を示す。\\
\begin{eqnarray}
T^{(0)}_0=2.0000000000000 誤差1.1415926535898\nonumber\\
T^{(0)}_1=2.7320508075689 誤差0.4095418460209\nonumber\\
T^{(0)}_2=2.9957090681024 誤差0.1458835854874\nonumber\\
T^{(0)}_3=3.0898191443572 誤差0.0517735092326\nonumber\\
T^{(0)}_4=3.1232530378277 誤差0.0183396157621\nonumber\\
T^{(1)}_1=2.9760677434252 誤差0.1655249101646\nonumber\\
T^{(1)}_2=3.0835951549470 誤差0.0579974986428\nonumber\\
T^{(1)}_3=3.1211891697754 誤差0.0204034838144\nonumber\\
T^{(1)}_4=3.1343976689846 誤差0.0071949846052\nonumber\\
T^{(2)}_2=3.0907636490484 誤差0.0508290045414\nonumber\\
T^{(2)}_3=3.1236954374306 誤差0.0178972161591\nonumber\\
T^{(2)}_4=3.1352782355985 誤差0.0063144179913\nonumber\\
T^{(3)}_3=3.1242181642304 誤差0.0173744893594\nonumber\\
T^{(3)}_4=3.1354620895377 誤差0.0061305640521\nonumber\\
T^{(4)}_4=3.1355061833625 誤差0.0060864702273\nonumber
\end{eqnarray}
\end{document}
