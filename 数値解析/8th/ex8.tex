\documentclass[]{jsarticle}
\usepackage{amsmath}
\usepackage{comment}
\title{第7回数値解析演習}
\author{081531257,早乙女 献自}
\date{2017/04/19}
\begin{document}
\maketitle
課題1 ガウスの消去法
\begin{eqnarray}
x_1-x_2+2x_3&=&4\nonumber\\
2x_1-2x_2+x_3&=&2\nonumber\\
-x_1+3x_2+x_3&=&8\nonumber
\end{eqnarray}
\noindent
(1)上記の式について手計算で計算した。その結果を以下に示す。\\
\begin{eqnarray}
\left[
\begin{array}{rrr|r}
  1&-1&2&4\\
   & & &\\
  2&-2&1&2\\
   & & &\\
  -1&3&1&8
\end{array}
\right]
&=&
\left[
\begin{array}{rrr|r}
  2&-2&1&2\\
   & & &\\
  1&-1&2&4\\
   & & &\\
  -1&3&1&8
\end{array}
\right]
\\\nonumber\\
&=&
\left[
\begin{array}{rrr|r}
  2&-2&1&2\\
   & & &\\
  0&0&\frac{3}{2}&3\\
   & & &\\
  0&2&\frac{3}{2}&9
\end{array}
\right]
\\\nonumber\\
&=&
\left[
\begin{array}{rrr|r}
  2&-2&1&2\\
   & & &\\
  0&2&\frac{3}{2}&9\\
   & & &\\
  0&0&\frac{3}{2}&3
\end{array}
\right]\\
\frac{3}{2}x_3&=&3\nonumber\\
x_3&=&2\nonumber\\
2x_2+3&=&9\nonumber\\
x_2&=&3\nonumber\\
2x_1-6+2&=&2\nonumber\\
x_1&=&3\nonumber
\end{eqnarray}

(1)では1列目最大の要素である2行目の2をピボットにするため、ピボッティングを行った。

(2)では1列目について前進消去を行った。

(3)では2列目について2行目以降で最大の要素である3行目の2をピボットにするためピボッティングを行った。この時点で下三角の要素についてすべて0になったため、前進消去を終了した。\\\\
(2)プログラムによりガウスの消去法を実装し、結果を出力した。実行結果を以下に示す。なお、下三角行列の値は使わないので、プログラム上では引き算を行っていない。\\
\$ ./a.out\\
2.000000 -2.000000 1.000000 \\
-1.000000 2.000000 1.500000 \\
1.000000 0.000000 1.500000 \\
x1 = 3.000000\\
x2 = 3.000000\\
x3 = 2.000000\\\\
課題2 ヒルベルト行列\\
ヒルベルト行列について、先ほど作成したガウスの消去法のプログラムで計算した。
そして、N次連立方程式についてNを1ずつ変えたところ、小数点以下7桁の精度でN=8において初めて誤差が生じた。その時の出力結果を示す。\\
\$ ./a.out\\
N=8\\
x1 = 1.0000000\\
x2 = 1.0000000\\
x3 = 1.0000000\\
x4 = 1.0000001\\
x5 = 0.9999997\\
x6 = 1.0000004\\
x7 = 0.9999997\\
x8 = 1.0000001
\end{document}
